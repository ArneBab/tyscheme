\scmkeyword{my-or}

\chapter{Macros}
\label{sugar}

\index{macro}
Users can create their own special forms by defining
{\em macros}.  A macro is a symbol that has a {\em
transformer procedure} associated with it.  When Scheme
encounters a macro-expression --- ie, a form whose
head is a macro ---, it applies the macro's transformer
to the subforms in the macro-expression, and evaluates
the result of the transformation.

Ideally, a macro specifies a purely textual
transformation from code text to other code text.  This
kind of transformation is useful for abbreviating an
involved and perhaps frequently occurring textual
pattern.

\scmkeyword{proc}

\index{define-macro@\q{define-macro}}
\index{when@\q{when}!macro for}

A macro is {\em defined} using the special form
\q{define-macro} (but see
sec~\ref{dialect-macro}).\f{MzScheme provides
\q{define-macro} via the \p{defmacro} library.  Use \q{(require (lib
"defmacro.ss"))} to load this library.}
For example, if your Scheme lacks the conditional
special form \q{when}, you could define
\q{when} as the following macro:

\q{
(define-macro when
  (lambda (test . branch)
    (list 'if test
      (cons 'begin branch))))
}


\n This defines a  \q{when}-transformer that would
convert a \q{when}-expression into the equivalent
\q{if}-expression.   With this macro definition in
place, the \q{when}-expression 

\q{
(when (< (pressure tube) 60)
   (open-valve tube)
   (attach floor-pump tube)
   (depress floor-pump 5)
   (detach floor-pump tube)
   (close-valve tube))
}

\n will be converted to another expression, the result
of applying the
\q{when}-transformer to the \q{when}-expression's
subforms:

\q{
(apply
  (lambda (test . branch)
    (list 'if test
      (cons 'begin branch)))
  '((< (pressure tube) 60)
      (open-valve tube)
      (attach floor-pump tube)
      (depress floor-pump 5)
      (detach floor-pump tube)
      (close-valve tube)))
}

\n The transformation yields the list

\q{
(if (< (pressure tube) 60)
    (begin
      (open-valve tube)
      (attach floor-pump tube)
      (depress floor-pump 5)
      (detach floor-pump tube)
      (close-valve tube)))
}

\n Scheme will then evaluate this expression, as it
  would any other.

\index{unless@\q{unless}!macro for}

As an additional example, here is the macro-definition
for \q{when}'s counterpart \q{unless}:

\q{
(define-macro unless
  (lambda (test . branch)
    (list 'if
          (list 'not test)
          (cons 'begin branch))))
}

\n Alternatively, we could invoke \q{when} inside
  \q{unless}'s definition:

\q{
(define-macro unless
  (lambda (test . branch)
    (cons 'when
          (cons (list 'not test) branch))))
}

\n Macro expansions can refer to other macros.

\section{Specifying the expansion as a template}

A macro transformer takes some s-expressions and
produces an s-expression that will be used as a form.
Typically this output is a list.  In our
\q{when} example, the output list is created using

\q{
(list 'if test
  (cons 'begin branch))
}

\n where \q{test} is bound to the macro's first
  subform, ie,

\q{
(< (pressure tube) 60)
}

\n and \q{branch} to the rest of the macro's subforms,
ie,

\q{
((open-valve tube)
 (attach floor-pump tube)
 (depress floor-pump 5)
 (detach floor-pump tube)
 (close-valve tube))
}

\index{`@\q{`} (backquote)}
\index{,@\q{,} (comma)}
\index{,"@@\q{,"@} (comma-splice)}

\n 
Output lists can be quite complicated.  It is easy to
see that a more ambitious macro than \q{when} could
lead to quite an elaborate construction process for the
output list.  In such cases, it is more convenient to
specify the macro's output form as a {\em template},
with the macro arguments inserted at appropriate places
to fill out the template for each particular use of the
macro.  Scheme provides the {\em backquote} syntax to
specify such templates.    Thus the expression

\q{
(list 'IF test
  (cons 'BEGIN branch))
}

\n is more conveniently written as

\q{
`(IF ,test
  (BEGIN ,@branch))
}

\n We can refashion the \q{when} macro-definition as:

\q{
(define-macro when
  (lambda (test . branch)
    `(IF ,test
         (BEGIN ,@branch))))
}

\n Note that the template format, unlike the earlier
list construction, gives immediate visual indication of
the shape of the output list.  The backquote (\q{`})
introduces a template for a list.  The elements of the
template appear {\em verbatim} in the resulting list,
{\em except} when they are prefixed by a {\em comma}
(`\q{,}') or a {\em comma-splice} (`\q{,@}').  (For the
purpose of illustration, we have written the verbatim
elements of the template in UPPER-CASE.)

The comma and the comma-splice are used to insert the
macro arguments into the template.  The comma inserts
the result of evaluating its following expression.  The
comma-splice inserts the result of evaluating its
following expression after {\em splicing} it, ie, it
removes the outermost set of parentheses.  (This
implies that an expression introduced by comma-splice
{\em must} be a list.)

In our example, given the values that \q{test} and
\q{branch} are bound to, it is easy to see that the
template will expand to the required

\q{
(IF (< (pressure tube) 60)
    (BEGIN
      (open-valve tube)
      (attach floor-pump tube)
      (depress floor-pump 5)
      (detach floor-pump tube)
      (close-valve tube)))
}

\section{Avoiding variable capture inside macros}

\index{macro!avoiding variable capture inside}
A two-argument disjunction form, \q{my-or}, could be
defined as follows:

\q{
(define-macro my-or
  (lambda (x y)
    `(if ,x ,x ,y)))
}

\n 
\q{my-or} takes two arguments and returns the value of
the first of them that is true (ie, non-\q{#f}).  In
particular, the second argument is evaluated only if
the first turns out to be false.

\q{
(my-or 1 2)
|evalsto 1

(my-or #f 2)
|evalsto 2
}

There is a problem with the \q{my-or} macro as it is
written.  It re-evaluates the first argument if it is
true: once in the \q{if}-test, and once again in the
``then'' branch.  This can cause undesired behavior if
the first argument were to contain side-effects, eg,

\q{
(my-or
  (begin 
    (display "doing first argument")
     (newline)
     #t)
  2)
}

\n displays \q{"doing first argument"} twice.

This can be avoided by storing
the \q{if}-test result in a local variable:

\q{
(define-macro my-or
  (lambda (x y)
    `(let ((temp ,x))
       (if temp temp ,y))))
}

\n This is almost OK, except in the case where the
  second argument happens to contain the same
  identifier \q{temp} as used in the macro definition.
  Eg,

\q{
(define temp 3)

(my-or #f temp)
|evalsto #f
}

\n Surely it should be 3!  The fiasco happens because
the macro uses a local variable \q{temp} to store the
value of the first argument (\q{#f}) and the 
variable 
\q{temp} in the second argument got {\em captured} by
the
\q{temp} introduced by the macro.

To avoid this, we need to be careful in choosing local
variables inside macro definitions.  We could choose
outlandish names for such variables and hope fervently
that nobody else comes up with them.   Eg,

\q{
(define-macro my-or
  (lambda (x y)
    `(let ((+temp ,x))
       (if +temp +temp ,y))))
}

\n This will work given the tacit understanding 
that \q{+temp} will not be used by code outside the
macro.  This is of course an understanding waiting to
be disillusioned.

\index{gensym@\q{gensym}}
\index{symbol!generated}

A more reliable, if verbose, approach is to use {\em
generated symbols} that are guaranteed not to be
obtainable by other means.  The procedure \q{gensym}
generates unique symbols each time it is called.  Here
is a safe definition for \q{my-or} using \q{gensym}:

\q{
(define-macro my-or
  (lambda (x y)
    (let ((temp (gensym)))
      `(let ((,temp ,x))
         (if ,temp ,temp ,y)))))
}

\n In the macros defined in this document, in order to be
concise, we will not use the \q{gensym} approach.
Instead, we will consider the point about variable
capture as having been made, and go ahead with the less
cluttered \q{+}-as-prefix approach.  We will leave it
to the astute reader to remember to convert these
\q{+}-identifiers into gensyms in the manner outlined
above.

\index{fluid-let@\q{fluid-let}!macro for}

\section{\q{fluid-let}}
\label{fluid-let-macro}

Here is a definition of a rather more complicated
macro, \q{fluid-let} (sec~\ref{fluid-let}).
\q{fluid-let} specifies temporary bindings for
a set of already existing lexical variables.  Given a
\q{fluid-let} expression such as

\q{
(fluid-let ((x 9) (y (+ y 1)))
  (+ x y))
}

\n we want the expansion to be

\q{
(let ((OLD-X x) (OLD-Y y))
  (set! x 9)
  (set! y (+ y 1))
  (let ((RESULT (begin (+ x y))))
    (set! x OLD-X)
    (set! y OLD-Y)
    RESULT))
}

\n where we want the identifiers \q{OLD-X}, \q{OLD-Y},
  and \q{RESULT} to be symbols that will not capture
variables in the expressions in the \q{fluid-let} form.

Here is how we go about fashioning a \q{fluid-let}
macro that implements what we want:

\q{
(define-macro fluid-let
  (lambda (xexe . body)
    (let ((xx (map car xexe))
          (ee (map cadr xexe))
          (old-xx (map (lambda (ig) (gensym)) xexe))
          (result (gensym)))
      `(let ,(map (lambda (old-x x) `(,old-x ,x)) 
                  old-xx xx)
         ,@(map (lambda (x e)
                  `(set! ,x ,e)) 
                xx ee)
         (let ((,result (begin ,@body)))
           ,@(map (lambda (x old-x)
                    `(set! ,x ,old-x)) 
                  xx old-xx)
           ,result)))))
}

\n The macro's arguments are:
\q{xexe}, the list of
variable/expression pairs introduced by the \q{fluid-let}; and  
\q{body}, the list of
expressions in the body of the \q{fluid-let}.  In our
example, these are \q{((x 9) (y (+ y 1)))} and \q{((+ x
y))} respectively.

The macro body introduces a bunch of local variables:
\q{xx} is the list of the variables extracted from the
variable/expression pairs. 
\q{ee} is the corresponding list of
expressions. \q{old-xx} is a list of fresh identifiers,
one for each variable in \q{xx}.  These are used to
store the {\em incoming} values of the \q{xx}, so we
can revert the \q{xx} back to them once the
\q{fluid-let} body has been evaluated.
\q{result} is another
fresh identifier, used to store the value of the
\q{fluid-let} body.  In our example, \q{xx} is \q{(x y)}
and \q{ee} is \q{(9 (+ y 1))}.  Depending on how your
system implements \q{gensym},
\q{old-xx} might be the
list \q{(GEN-63 GEN-64)}, and \q{result} might be \q{GEN-65}.

The output list is created by the macro for our given
example looks like

\q{
(let ((GEN-63 x) (GEN-64 y))
  (set! x 9)
  (set! y (+ y 1))
  (let ((GEN-65 (begin (+ x y))))
    (set! x GEN-63)
    (set! y GEN-64)
    GEN-65))
}

\n 
which matches our requirement.

